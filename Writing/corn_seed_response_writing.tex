% Options for packages loaded elsewhere
\PassOptionsToPackage{unicode}{hyperref}
\PassOptionsToPackage{hyphens}{url}
\PassOptionsToPackage{dvipsnames,svgnames,x11names}{xcolor}
%
\documentclass[
]{article}

\usepackage{amsmath,amssymb}
\usepackage{iftex}
\ifPDFTeX
  \usepackage[T1]{fontenc}
  \usepackage[utf8]{inputenc}
  \usepackage{textcomp} % provide euro and other symbols
\else % if luatex or xetex
  \usepackage{unicode-math}
  \defaultfontfeatures{Scale=MatchLowercase}
  \defaultfontfeatures[\rmfamily]{Ligatures=TeX,Scale=1}
\fi
\usepackage[]{times}
\ifPDFTeX\else  
    % xetex/luatex font selection
\fi
% Use upquote if available, for straight quotes in verbatim environments
\IfFileExists{upquote.sty}{\usepackage{upquote}}{}
\IfFileExists{microtype.sty}{% use microtype if available
  \usepackage[]{microtype}
  \UseMicrotypeSet[protrusion]{basicmath} % disable protrusion for tt fonts
}{}
\makeatletter
\@ifundefined{KOMAClassName}{% if non-KOMA class
  \IfFileExists{parskip.sty}{%
    \usepackage{parskip}
  }{% else
    \setlength{\parindent}{0pt}
    \setlength{\parskip}{6pt plus 2pt minus 1pt}}
}{% if KOMA class
  \KOMAoptions{parskip=half}}
\makeatother
\usepackage{xcolor}
\setlength{\emergencystretch}{3em} % prevent overfull lines
\setcounter{secnumdepth}{2}
% Make \paragraph and \subparagraph free-standing
\makeatletter
\ifx\paragraph\undefined\else
  \let\oldparagraph\paragraph
  \renewcommand{\paragraph}{
    \@ifstar
      \xxxParagraphStar
      \xxxParagraphNoStar
  }
  \newcommand{\xxxParagraphStar}[1]{\oldparagraph*{#1}\mbox{}}
  \newcommand{\xxxParagraphNoStar}[1]{\oldparagraph{#1}\mbox{}}
\fi
\ifx\subparagraph\undefined\else
  \let\oldsubparagraph\subparagraph
  \renewcommand{\subparagraph}{
    \@ifstar
      \xxxSubParagraphStar
      \xxxSubParagraphNoStar
  }
  \newcommand{\xxxSubParagraphStar}[1]{\oldsubparagraph*{#1}\mbox{}}
  \newcommand{\xxxSubParagraphNoStar}[1]{\oldsubparagraph{#1}\mbox{}}
\fi
\makeatother


\providecommand{\tightlist}{%
  \setlength{\itemsep}{0pt}\setlength{\parskip}{0pt}}\usepackage{longtable,booktabs,array}
\usepackage{calc} % for calculating minipage widths
% Correct order of tables after \paragraph or \subparagraph
\usepackage{etoolbox}
\makeatletter
\patchcmd\longtable{\par}{\if@noskipsec\mbox{}\fi\par}{}{}
\makeatother
% Allow footnotes in longtable head/foot
\IfFileExists{footnotehyper.sty}{\usepackage{footnotehyper}}{\usepackage{footnote}}
\makesavenoteenv{longtable}
\usepackage{graphicx}
\makeatletter
\def\maxwidth{\ifdim\Gin@nat@width>\linewidth\linewidth\else\Gin@nat@width\fi}
\def\maxheight{\ifdim\Gin@nat@height>\textheight\textheight\else\Gin@nat@height\fi}
\makeatother
% Scale images if necessary, so that they will not overflow the page
% margins by default, and it is still possible to overwrite the defaults
% using explicit options in \includegraphics[width, height, ...]{}
\setkeys{Gin}{width=\maxwidth,height=\maxheight,keepaspectratio}
% Set default figure placement to htbp
\makeatletter
\def\fps@figure{htbp}
\makeatother
% definitions for citeproc citations
\NewDocumentCommand\citeproctext{}{}
\NewDocumentCommand\citeproc{mm}{%
  \begingroup\def\citeproctext{#2}\cite{#1}\endgroup}
\makeatletter
 % allow citations to break across lines
 \let\@cite@ofmt\@firstofone
 % avoid brackets around text for \cite:
 \def\@biblabel#1{}
 \def\@cite#1#2{{#1\if@tempswa , #2\fi}}
\makeatother
\newlength{\cslhangindent}
\setlength{\cslhangindent}{1.5em}
\newlength{\csllabelwidth}
\setlength{\csllabelwidth}{3em}
\newenvironment{CSLReferences}[2] % #1 hanging-indent, #2 entry-spacing
 {\begin{list}{}{%
  \setlength{\itemindent}{0pt}
  \setlength{\leftmargin}{0pt}
  \setlength{\parsep}{0pt}
  % turn on hanging indent if param 1 is 1
  \ifodd #1
   \setlength{\leftmargin}{\cslhangindent}
   \setlength{\itemindent}{-1\cslhangindent}
  \fi
  % set entry spacing
  \setlength{\itemsep}{#2\baselineskip}}}
 {\end{list}}
\usepackage{calc}
\newcommand{\CSLBlock}[1]{\hfill\break\parbox[t]{\linewidth}{\strut\ignorespaces#1\strut}}
\newcommand{\CSLLeftMargin}[1]{\parbox[t]{\csllabelwidth}{\strut#1\strut}}
\newcommand{\CSLRightInline}[1]{\parbox[t]{\linewidth - \csllabelwidth}{\strut#1\strut}}
\newcommand{\CSLIndent}[1]{\hspace{\cslhangindent}#1}

% Place figures and tables exactly where they were called
\usepackage{float}
\floatplacement{figure}{H}
\floatplacement{table}{H}

% Recommended by the modelsummary package
\usepackage{booktabs}
\usepackage{siunitx}
\newcolumntype{d}{S[input-symbols = ()]}

% Add affiliations (title.tex needs to be called under template-partials)
\usepackage[noblocks]{authblk}
\renewcommand*{\Authsep}{, }
\renewcommand*{\Authand}{, }
\renewcommand*{\Authands}{, }
\renewcommand\Affilfont{\small}

% Add line numbers
\usepackage{lineno}
\linenumbers
\usepackage{fontspec}
\usepackage{multirow}
\usepackage{multicol}
\usepackage{colortbl}
\usepackage{hhline}
\newlength\Oldarrayrulewidth
\newlength\Oldtabcolsep
\usepackage{longtable}
\usepackage{array}
\usepackage{hyperref}
\usepackage{float}
\usepackage{wrapfig}
\makeatletter
\@ifpackageloaded{caption}{}{\usepackage{caption}}
\AtBeginDocument{%
\ifdefined\contentsname
  \renewcommand*\contentsname{Table of contents}
\else
  \newcommand\contentsname{Table of contents}
\fi
\ifdefined\listfigurename
  \renewcommand*\listfigurename{List of Figures}
\else
  \newcommand\listfigurename{List of Figures}
\fi
\ifdefined\listtablename
  \renewcommand*\listtablename{List of Tables}
\else
  \newcommand\listtablename{List of Tables}
\fi
\ifdefined\figurename
  \renewcommand*\figurename{Figure}
\else
  \newcommand\figurename{Figure}
\fi
\ifdefined\tablename
  \renewcommand*\tablename{Table}
\else
  \newcommand\tablename{Table}
\fi
}
\@ifpackageloaded{float}{}{\usepackage{float}}
\floatstyle{ruled}
\@ifundefined{c@chapter}{\newfloat{codelisting}{h}{lop}}{\newfloat{codelisting}{h}{lop}[chapter]}
\floatname{codelisting}{Listing}
\newcommand*\listoflistings{\listof{codelisting}{List of Listings}}
\makeatother
\makeatletter
\makeatother
\makeatletter
\@ifpackageloaded{caption}{}{\usepackage{caption}}
\@ifpackageloaded{subcaption}{}{\usepackage{subcaption}}
\makeatother

\ifLuaTeX
  \usepackage{selnolig}  % disable illegal ligatures
\fi
\usepackage{bookmark}

\IfFileExists{xurl.sty}{\usepackage{xurl}}{} % add URL line breaks if available
\urlstyle{same} % disable monospaced font for URLs
\hypersetup{
  pdftitle={Evaluating the Profitability of Corn Seeding Decisions: ~Insights from On-Farm Precision Experiments Data},
  pdfauthor={Jaeseok Hwang; Taro Mieno; David S Bullock},
  pdfkeywords={EOSR, OFPE, Yield Seed Response},
  colorlinks=true,
  linkcolor={blue},
  filecolor={Maroon},
  citecolor={red},
  urlcolor={Blue},
  pdfcreator={LaTeX via pandoc}}


\title{Evaluating the Profitability of Corn Seeding Decisions: ~Insights
from On-Farm Precision Experiments Data}


  \author{Jaeseok Hwang}
            \affil{%
                  University of Illinois at Urbana Champaign
              }
        \author{Taro Mieno}
            \affil{%
                  University of Nebraska-Lincoln
              }
        \author{David S Bullock}
            \affil{%
                  University of Illinois at Urbana Champaign
              }
      
\date{2024-10-13}
\begin{document}
\maketitle
\begin{abstract}
Efficient input use is a critical challenge in modern agriculture,
particularly as farmers seek to balance productivity with cost
management. In many of U.S. corn fields, farmers often apply seeding
rates based on their own historical practices rather than data-driven
economic optimization, leading to potential inaccurate input
application. This research addresses the question of how profitable
current corn seeding decisions are and whether farmers could increase
profitability by estimating yield seed(S) response and Economically
Otimal Seeding Rates (EOSR) with the On-Farm Precision Experiment
(OFPE). Using data from 97 OFPE trials conducted between 2016 and 2023,
this study contrasts farmers' status quo seeding rates (SQSR) with EOSR
estimates derived from Generalized Additive Model (GAM) regression.
Results indicate that, on average, farmers overapply by 3.8K seeds per
acre, leading to an average loss of \$24.7 per acre in 40 percent of the
trials. The analysis provides evidence for high-rate seeding practices
to enhance profitability, with potential implications for agricultural
policy or extension.
\end{abstract}


\newpage

\section{Introduction}\label{introduction}

From the early 1970s, for about 50 years, maize yield in U.S. has been
gradually increasing and it almost doubled, from 91.3 in 1973 to 177.3
bushels per acre in 2023 (NASS (\citeproc{ref-usdanass}{2024})). This
considerable growth in yield attributes to innovations in genotype,
environment and management but still it is not clear how much each
factors contribute to yield increase since their interactions in yield
responses are very complicated(Morris et al.
(\citeproc{ref-morris2018strengths}{2018})). The main drivers of the
yield increase are, however, the improvement of genetically engineered
seed and hybrid seed. These innovations in seed technology enables
higher density of seed population endure stress of competition within a
given area of planting and it drastically increases the probability of
germination (Fernandez-Cornejo et al.
(\citeproc{ref-fernandez2014genetically}{2014})). Therefore, raising the
seeding rate with the improved seed varieties promote the per acre corn
yield to the current level.

The increase in yield enhances the net revenue of farmer, with the
relatively cheap cost of seed, while the seeding rate has increased
during the late 20th century for 30years, from 1970s to 1990s. However,
from the 2000s, there has been a fluctuation in seed cost and the
portion of the seed cost in the total operation cost increased a lot
(Saavoss et al. (\citeproc{ref-saavoss2021trends}{2021})). As a result,
the portion of total seed cost and fertilizer cost in the operation cost
became much closer, and the benefit of the estimated Economically
Optimum Seeding Rate (EOSR) has been increased in terms of saving
operation cost and enhancing total revenue of corn production.

\begin{figure}

\centering{

\includegraphics[width=1\textwidth,height=\textheight]{corn_seed_response_writing_files/figure-pdf/fig-rev_seed_comb-1.pdf}

}

\caption{\label{fig-rev_seed_comb}Net revenue after seed cost figure
(1996 to 2023)}

\end{figure}%

However, despite of this increasing burden of seed cost, farmers do not
frequently adjust their seeding rates with respect to their very recent
variation in market price and weather condition(NASS
(\citeproc{ref-usdanass}{2024})). Also, NASS
(\citeproc{ref-usdanass}{2024}) data shows the evidence that the farmers
do not really adjust their seeding rate during the time that the corn
price decreases, while the seed cost increases and per acre revenue
decreases. Figure~\ref{fig-rev_seed_comb} shows the trend of the yearly
changes of net revenue after seed cost over the recent 30 years. The
left plot shows the continuous increase in net revenue over 30 years,
however, when we adjust it by Producer Price Index(PPI) of corn in
agricultural commodity sector, the net revenue stops increasing from
2016 and there are high fluctuation and decreasing trend of net revenue
very recently.

In various recent resources, the estimated EOSR on the midwest Corn-belt
(Heartland) are ranged from 32k to 36k (Nafziger and Fontes
(\citeproc{ref-illinois2023seed}{2023}),Licht, Lenssen, and Elmore
(\citeproc{ref-licht2017corn}{2017}),Lindsey, Thomison, and Nafziger
(\citeproc{ref-lindsey2018modeling}{2018}),Nielsen et al.
(\citeproc{ref-nielsen2019yield}{2019}),Lacasa et al.
(\citeproc{ref-lacasa2020bayesian}{2020})). For instance, Assefa et al.
(\citeproc{ref-assefa2018analysis}{2018}) estimated EOSR with the 14
years of on-farm experimental results from 22 different states, and it
recommend 34K as a EOSR in the moderate weather condition in the Midwest
corn-belt. However, the impact of recently increased draught and extreme
weather decreases the probability of high attainable yield in many of
the Midwest fields and it doubt the profitability of the aforementioned
high rates, 34K, seeding (Kukal and Irmak
(\citeproc{ref-kukal2018climate}{2018}),Rigden et al.
(\citeproc{ref-rigden2020combined}{2020})).

This research, hence, investigate how much the farmer's choice of corn
seeding rate are profitable with the recent empirical On-Farm Precision
Experiment(OFPE) data which are collected from 2016 to 2023 over 8
different states in U.S. To evaluate the profit of farmer's status quo
seeding rate (SQSR), yield seed(S) response function for each experiment
fields are calculated by the Generalized Additive Model(GAM) regression.
Then, the estimated profits of SQSRs and estimated EOSRs are evaluated
by the type of yield S response and seeding rate differences in SQSR and
EOSR.

The result find out the evidence that the farmers are likely to plant
about 3.8K more seed than estimated EOSR, and at the 40 out of 100
participated trials, farmers loss about \$24.7 per acre potential profit
due to excessive high seeding decision behavior.

\section{Method}\label{method}

\subsection{Datasets}\label{datasets}

This research mainly evaluates the estimated profit of farmer's SQSR and
EOSR by projecting it on the given climate conditions. To estimate the
profit at the SQSR and EOSR accurately, it is requisite to estimate
field specific yield S response function. Thus, to estimate yield S
response of the experimental fields, OFPE data were collected and
processed by the following steps.

First, 163 OFPE data was adopted from the database that is collected by
the Data Intensive Farm Management (DIFM) project Bullock et al.
(\citeproc{ref-bullock2019data}{2019}). The 169 dataset was gathered
from the 42 farms which are located on 8 differents state of Midwest
Corn-belt. DIFM project consults OFPE by designing S x N trial input
combination to be applied and planted into trial polygon, and it
prevents seed and nitrogen having spatial correlation. Also, these two
controlled inputs are spatially independent with soil and field specific
characteristics Li, Taro Mieno, and Bullock
(\citeproc{ref-lix2021design}{2021}).

\begin{figure}

\centering{

\includegraphics[width=0.6\textwidth,height=\textheight]{corn_seed_response_writing_files/figure-pdf/fig-td_sample-1.pdf}

}

\caption{\label{fig-td_sample}On Farm Trial Design Sample}

\end{figure}%

Figure~\ref{fig-td_sample} shows the example of the design, and the
range of trial inputs are determined by assigning farmer's SQSR into the
middle of the trial inputs range. Following this trial-design, farmers
apply the assigned rates and harvest the crop by using GPS-linked
vehicle, and it records the S , N and yield data in real-time. The
experimental data, yield, S and N are cleaned and processed by the
protocol in Edge, Mieno, and Bullock
(\citeproc{ref-edge2024processing}{2024}). The protocol creates yield
polygon by eliminating the highly deviated or the misaligned yield
points. The size of the yield polygon is determined by the size of the
trial polygon, swath-width and distance of the harvester, applicator and
planter. Input polygons for S and N are created by removing outliers and
the data points which are located in the transition zone where the
vehicle changes the trial rate. After this individual cleaning process,
it calculates the median value of input polygons into yield polygons to
combine yield and input polygon. At this process, the yield polygon
where the combined input polygon have high deviation are removed to
prevent input straddling problem. Through this cleaning protocol, 66 out
of 163 OFPE data are excluded in the dataset since they have too small
observations due to straddling problem or errors in the field-collected
raw level data.

::: \{\#tbl-dat\_summary, tbl-cap: ``Data summary by year (2016 to
2023)'' .cell layout-align=``center''\} ::: \{.cell-output-display\}

\global\setlength{\Oldarrayrulewidth}{\arrayrulewidth}

\global\setlength{\Oldtabcolsep}{\tabcolsep}

\setlength{\tabcolsep}{2pt}

\renewcommand*{\arraystretch}{1.5}



\providecommand{\ascline}[3]{\noalign{\global\arrayrulewidth #1}\arrayrulecolor[HTML]{#2}\cline{#3}}

\begin{longtable*}[c]{|p{0.40in}|p{0.50in}|p{0.40in}|p{0.40in}|p{0.50in}|p{0.70in}|p{0.70in}|p{0.70in}|p{0.70in}}



\ascline{1.5pt}{666666}{1-9}

\multicolumn{1}{>{\centering}m{\dimexpr 0.4in+0\tabcolsep}}{\textcolor[HTML]{000000}{\fontsize{8}{8}\selectfont{Year}}} & \multicolumn{1}{>{\centering}m{\dimexpr 0.5in+0\tabcolsep}}{\textcolor[HTML]{000000}{\fontsize{8}{8}\selectfont{Field}}\textcolor[HTML]{000000}{\fontsize{8}{8}\selectfont{\linebreak }}\textcolor[HTML]{000000}{\fontsize{8}{8}\selectfont{Count}}} & \multicolumn{1}{>{\centering}m{\dimexpr 0.4in+0\tabcolsep}}{\textcolor[HTML]{000000}{\fontsize{8}{8}\selectfont{Mean}}\textcolor[HTML]{000000}{\fontsize{8}{8}\selectfont{\linebreak }}\textcolor[HTML]{000000}{\fontsize{8}{8}\selectfont{Yield}}\textcolor[HTML]{000000}{\fontsize{8}{8}\selectfont{\linebreak }}\textcolor[HTML]{000000}{\fontsize{8}{8}\selectfont{(bu/ac)}}} & \multicolumn{1}{>{\centering}m{\dimexpr 0.4in+0\tabcolsep}}{\textcolor[HTML]{000000}{\fontsize{8}{8}\selectfont{SQSR}}} & \multicolumn{1}{>{\centering}m{\dimexpr 0.5in+0\tabcolsep}}{\textcolor[HTML]{000000}{\fontsize{8}{8}\selectfont{USDASR}}} & \multicolumn{1}{>{\centering}m{\dimexpr 0.7in+0\tabcolsep}}{\textcolor[HTML]{000000}{\fontsize{8}{8}\selectfont{Precipitation}}\textcolor[HTML]{000000}{\fontsize{8}{8}\selectfont{\linebreak }}\textcolor[HTML]{000000}{\fontsize{8}{8}\selectfont{(In-Season)}}} & \multicolumn{1}{>{\centering}m{\dimexpr 0.7in+0\tabcolsep}}{\textcolor[HTML]{000000}{\fontsize{8}{8}\selectfont{GDD}}\textcolor[HTML]{000000}{\fontsize{8}{8}\selectfont{\linebreak }}\textcolor[HTML]{000000}{\fontsize{8}{8}\selectfont{(In-Season)}}} & \multicolumn{1}{>{\centering}m{\dimexpr 0.7in+0\tabcolsep}}{\textcolor[HTML]{000000}{\fontsize{8}{8}\selectfont{Precipitation}}\textcolor[HTML]{000000}{\fontsize{8}{8}\selectfont{\linebreak }}\textcolor[HTML]{000000}{\fontsize{8}{8}\selectfont{(30Year)}}} & \multicolumn{1}{>{\centering}m{\dimexpr 0.7in+0\tabcolsep}}{\textcolor[HTML]{000000}{\fontsize{8}{8}\selectfont{GDD}}\textcolor[HTML]{000000}{\fontsize{8}{8}\selectfont{\linebreak }}\textcolor[HTML]{000000}{\fontsize{8}{8}\selectfont{(30Year)}}} \\

\ascline{1.5pt}{666666}{1-9}\endfirsthead 

\ascline{1.5pt}{666666}{1-9}

\multicolumn{1}{>{\centering}m{\dimexpr 0.4in+0\tabcolsep}}{\textcolor[HTML]{000000}{\fontsize{8}{8}\selectfont{Year}}} & \multicolumn{1}{>{\centering}m{\dimexpr 0.5in+0\tabcolsep}}{\textcolor[HTML]{000000}{\fontsize{8}{8}\selectfont{Field}}\textcolor[HTML]{000000}{\fontsize{8}{8}\selectfont{\linebreak }}\textcolor[HTML]{000000}{\fontsize{8}{8}\selectfont{Count}}} & \multicolumn{1}{>{\centering}m{\dimexpr 0.4in+0\tabcolsep}}{\textcolor[HTML]{000000}{\fontsize{8}{8}\selectfont{Mean}}\textcolor[HTML]{000000}{\fontsize{8}{8}\selectfont{\linebreak }}\textcolor[HTML]{000000}{\fontsize{8}{8}\selectfont{Yield}}\textcolor[HTML]{000000}{\fontsize{8}{8}\selectfont{\linebreak }}\textcolor[HTML]{000000}{\fontsize{8}{8}\selectfont{(bu/ac)}}} & \multicolumn{1}{>{\centering}m{\dimexpr 0.4in+0\tabcolsep}}{\textcolor[HTML]{000000}{\fontsize{8}{8}\selectfont{SQSR}}} & \multicolumn{1}{>{\centering}m{\dimexpr 0.5in+0\tabcolsep}}{\textcolor[HTML]{000000}{\fontsize{8}{8}\selectfont{USDASR}}} & \multicolumn{1}{>{\centering}m{\dimexpr 0.7in+0\tabcolsep}}{\textcolor[HTML]{000000}{\fontsize{8}{8}\selectfont{Precipitation}}\textcolor[HTML]{000000}{\fontsize{8}{8}\selectfont{\linebreak }}\textcolor[HTML]{000000}{\fontsize{8}{8}\selectfont{(In-Season)}}} & \multicolumn{1}{>{\centering}m{\dimexpr 0.7in+0\tabcolsep}}{\textcolor[HTML]{000000}{\fontsize{8}{8}\selectfont{GDD}}\textcolor[HTML]{000000}{\fontsize{8}{8}\selectfont{\linebreak }}\textcolor[HTML]{000000}{\fontsize{8}{8}\selectfont{(In-Season)}}} & \multicolumn{1}{>{\centering}m{\dimexpr 0.7in+0\tabcolsep}}{\textcolor[HTML]{000000}{\fontsize{8}{8}\selectfont{Precipitation}}\textcolor[HTML]{000000}{\fontsize{8}{8}\selectfont{\linebreak }}\textcolor[HTML]{000000}{\fontsize{8}{8}\selectfont{(30Year)}}} & \multicolumn{1}{>{\centering}m{\dimexpr 0.7in+0\tabcolsep}}{\textcolor[HTML]{000000}{\fontsize{8}{8}\selectfont{GDD}}\textcolor[HTML]{000000}{\fontsize{8}{8}\selectfont{\linebreak }}\textcolor[HTML]{000000}{\fontsize{8}{8}\selectfont{(30Year)}}} \\

\ascline{1.5pt}{666666}{1-9}\endhead



\multicolumn{1}{>{\centering}m{\dimexpr 0.4in+0\tabcolsep}}{\textcolor[HTML]{000000}{\fontsize{8}{8}\selectfont{2,016}}} & \multicolumn{1}{>{\centering}m{\dimexpr 0.5in+0\tabcolsep}}{\textcolor[HTML]{000000}{\fontsize{8}{8}\selectfont{4}}} & \multicolumn{1}{>{\centering}m{\dimexpr 0.4in+0\tabcolsep}}{\textcolor[HTML]{000000}{\fontsize{8}{8}\selectfont{178.9}}\textcolor[HTML]{000000}{\fontsize{8}{8}\selectfont{\linebreak }}\textcolor[HTML]{000000}{\fontsize{8}{8}\selectfont{(59.7)}}} & \multicolumn{1}{>{\centering}m{\dimexpr 0.4in+0\tabcolsep}}{\textcolor[HTML]{000000}{\fontsize{8}{8}\selectfont{35.5}}\textcolor[HTML]{000000}{\fontsize{8}{8}\selectfont{\linebreak }}\textcolor[HTML]{000000}{\fontsize{8}{8}\selectfont{(0.6)}}} & \multicolumn{1}{>{\centering}m{\dimexpr 0.5in+0\tabcolsep}}{\textcolor[HTML]{000000}{\fontsize{8}{8}\selectfont{31.1}}\textcolor[HTML]{000000}{\fontsize{8}{8}\selectfont{\linebreak }}\textcolor[HTML]{000000}{\fontsize{8}{8}\selectfont{(0.0)}}} & \multicolumn{1}{>{\centering}m{\dimexpr 0.7in+0\tabcolsep}}{\textcolor[HTML]{000000}{\fontsize{8}{8}\selectfont{787.7}}\textcolor[HTML]{000000}{\fontsize{8}{8}\selectfont{\linebreak }}\textcolor[HTML]{000000}{\fontsize{8}{8}\selectfont{(38.0)}}} & \multicolumn{1}{>{\centering}m{\dimexpr 0.7in+0\tabcolsep}}{\textcolor[HTML]{000000}{\fontsize{8}{8}\selectfont{2000.5}}\textcolor[HTML]{000000}{\fontsize{8}{8}\selectfont{\linebreak }}\textcolor[HTML]{000000}{\fontsize{8}{8}\selectfont{(95.3)}}} & \multicolumn{1}{>{\centering}m{\dimexpr 0.7in+0\tabcolsep}}{\textcolor[HTML]{000000}{\fontsize{8}{8}\selectfont{646.3}}\textcolor[HTML]{000000}{\fontsize{8}{8}\selectfont{\linebreak }}\textcolor[HTML]{000000}{\fontsize{8}{8}\selectfont{(26.8)}}} & \multicolumn{1}{>{\centering}m{\dimexpr 0.7in+0\tabcolsep}}{\textcolor[HTML]{000000}{\fontsize{8}{8}\selectfont{1871.0}}\textcolor[HTML]{000000}{\fontsize{8}{8}\selectfont{\linebreak }}\textcolor[HTML]{000000}{\fontsize{8}{8}\selectfont{(81.4)}}} \\





\multicolumn{1}{>{\centering}m{\dimexpr 0.4in+0\tabcolsep}}{\textcolor[HTML]{000000}{\fontsize{8}{8}\selectfont{2,017}}} & \multicolumn{1}{>{\centering}m{\dimexpr 0.5in+0\tabcolsep}}{\textcolor[HTML]{000000}{\fontsize{8}{8}\selectfont{6}}} & \multicolumn{1}{>{\centering}m{\dimexpr 0.4in+0\tabcolsep}}{\textcolor[HTML]{000000}{\fontsize{8}{8}\selectfont{227.0}}\textcolor[HTML]{000000}{\fontsize{8}{8}\selectfont{\linebreak }}\textcolor[HTML]{000000}{\fontsize{8}{8}\selectfont{(29.3)}}} & \multicolumn{1}{>{\centering}m{\dimexpr 0.4in+0\tabcolsep}}{\textcolor[HTML]{000000}{\fontsize{8}{8}\selectfont{34.5}}\textcolor[HTML]{000000}{\fontsize{8}{8}\selectfont{\linebreak }}\textcolor[HTML]{000000}{\fontsize{8}{8}\selectfont{(1.4)}}} & \multicolumn{1}{>{\centering}m{\dimexpr 0.5in+0\tabcolsep}}{\textcolor[HTML]{000000}{\fontsize{8}{8}\selectfont{30.6}}\textcolor[HTML]{000000}{\fontsize{8}{8}\selectfont{\linebreak }}\textcolor[HTML]{000000}{\fontsize{8}{8}\selectfont{(0.8)}}} & \multicolumn{1}{>{\centering}m{\dimexpr 0.7in+0\tabcolsep}}{\textcolor[HTML]{000000}{\fontsize{8}{8}\selectfont{611.9}}\textcolor[HTML]{000000}{\fontsize{8}{8}\selectfont{\linebreak }}\textcolor[HTML]{000000}{\fontsize{8}{8}\selectfont{(65.9)}}} & \multicolumn{1}{>{\centering}m{\dimexpr 0.7in+0\tabcolsep}}{\textcolor[HTML]{000000}{\fontsize{8}{8}\selectfont{1825.6}}\textcolor[HTML]{000000}{\fontsize{8}{8}\selectfont{\linebreak }}\textcolor[HTML]{000000}{\fontsize{8}{8}\selectfont{(159.4)}}} & \multicolumn{1}{>{\centering}m{\dimexpr 0.7in+0\tabcolsep}}{\textcolor[HTML]{000000}{\fontsize{8}{8}\selectfont{631.4}}\textcolor[HTML]{000000}{\fontsize{8}{8}\selectfont{\linebreak }}\textcolor[HTML]{000000}{\fontsize{8}{8}\selectfont{(26.0)}}} & \multicolumn{1}{>{\centering}m{\dimexpr 0.7in+0\tabcolsep}}{\textcolor[HTML]{000000}{\fontsize{8}{8}\selectfont{1801.7}}\textcolor[HTML]{000000}{\fontsize{8}{8}\selectfont{\linebreak }}\textcolor[HTML]{000000}{\fontsize{8}{8}\selectfont{(152.0)}}} \\





\multicolumn{1}{>{\centering}m{\dimexpr 0.4in+0\tabcolsep}}{\textcolor[HTML]{000000}{\fontsize{8}{8}\selectfont{2,018}}} & \multicolumn{1}{>{\centering}m{\dimexpr 0.5in+0\tabcolsep}}{\textcolor[HTML]{000000}{\fontsize{8}{8}\selectfont{12}}} & \multicolumn{1}{>{\centering}m{\dimexpr 0.4in+0\tabcolsep}}{\textcolor[HTML]{000000}{\fontsize{8}{8}\selectfont{231.8}}\textcolor[HTML]{000000}{\fontsize{8}{8}\selectfont{\linebreak }}\textcolor[HTML]{000000}{\fontsize{8}{8}\selectfont{(23.6)}}} & \multicolumn{1}{>{\centering}m{\dimexpr 0.4in+0\tabcolsep}}{\textcolor[HTML]{000000}{\fontsize{8}{8}\selectfont{35.0}}\textcolor[HTML]{000000}{\fontsize{8}{8}\selectfont{\linebreak }}\textcolor[HTML]{000000}{\fontsize{8}{8}\selectfont{(1.0)}}} & \multicolumn{1}{>{\centering}m{\dimexpr 0.5in+0\tabcolsep}}{\textcolor[HTML]{000000}{\fontsize{8}{8}\selectfont{31.5}}\textcolor[HTML]{000000}{\fontsize{8}{8}\selectfont{\linebreak }}\textcolor[HTML]{000000}{\fontsize{8}{8}\selectfont{(0.8)}}} & \multicolumn{1}{>{\centering}m{\dimexpr 0.7in+0\tabcolsep}}{\textcolor[HTML]{000000}{\fontsize{8}{8}\selectfont{656.5}}\textcolor[HTML]{000000}{\fontsize{8}{8}\selectfont{\linebreak }}\textcolor[HTML]{000000}{\fontsize{8}{8}\selectfont{(100.2)}}} & \multicolumn{1}{>{\centering}m{\dimexpr 0.7in+0\tabcolsep}}{\textcolor[HTML]{000000}{\fontsize{8}{8}\selectfont{1919.5}}\textcolor[HTML]{000000}{\fontsize{8}{8}\selectfont{\linebreak }}\textcolor[HTML]{000000}{\fontsize{8}{8}\selectfont{(169.9)}}} & \multicolumn{1}{>{\centering}m{\dimexpr 0.7in+0\tabcolsep}}{\textcolor[HTML]{000000}{\fontsize{8}{8}\selectfont{630.4}}\textcolor[HTML]{000000}{\fontsize{8}{8}\selectfont{\linebreak }}\textcolor[HTML]{000000}{\fontsize{8}{8}\selectfont{(29.6)}}} & \multicolumn{1}{>{\centering}m{\dimexpr 0.7in+0\tabcolsep}}{\textcolor[HTML]{000000}{\fontsize{8}{8}\selectfont{1710.9}}\textcolor[HTML]{000000}{\fontsize{8}{8}\selectfont{\linebreak }}\textcolor[HTML]{000000}{\fontsize{8}{8}\selectfont{(179.7)}}} \\





\multicolumn{1}{>{\centering}m{\dimexpr 0.4in+0\tabcolsep}}{\textcolor[HTML]{000000}{\fontsize{8}{8}\selectfont{2,019}}} & \multicolumn{1}{>{\centering}m{\dimexpr 0.5in+0\tabcolsep}}{\textcolor[HTML]{000000}{\fontsize{8}{8}\selectfont{8}}} & \multicolumn{1}{>{\centering}m{\dimexpr 0.4in+0\tabcolsep}}{\textcolor[HTML]{000000}{\fontsize{8}{8}\selectfont{186.4}}\textcolor[HTML]{000000}{\fontsize{8}{8}\selectfont{\linebreak }}\textcolor[HTML]{000000}{\fontsize{8}{8}\selectfont{(27.2)}}} & \multicolumn{1}{>{\centering}m{\dimexpr 0.4in+0\tabcolsep}}{\textcolor[HTML]{000000}{\fontsize{8}{8}\selectfont{33.9}}\textcolor[HTML]{000000}{\fontsize{8}{8}\selectfont{\linebreak }}\textcolor[HTML]{000000}{\fontsize{8}{8}\selectfont{(2.1)}}} & \multicolumn{1}{>{\centering}m{\dimexpr 0.5in+0\tabcolsep}}{\textcolor[HTML]{000000}{\fontsize{8}{8}\selectfont{30.8}}\textcolor[HTML]{000000}{\fontsize{8}{8}\selectfont{\linebreak }}\textcolor[HTML]{000000}{\fontsize{8}{8}\selectfont{(0.3)}}} & \multicolumn{1}{>{\centering}m{\dimexpr 0.7in+0\tabcolsep}}{\textcolor[HTML]{000000}{\fontsize{8}{8}\selectfont{776.6}}\textcolor[HTML]{000000}{\fontsize{8}{8}\selectfont{\linebreak }}\textcolor[HTML]{000000}{\fontsize{8}{8}\selectfont{(117.7)}}} & \multicolumn{1}{>{\centering}m{\dimexpr 0.7in+0\tabcolsep}}{\textcolor[HTML]{000000}{\fontsize{8}{8}\selectfont{1791.3}}\textcolor[HTML]{000000}{\fontsize{8}{8}\selectfont{\linebreak }}\textcolor[HTML]{000000}{\fontsize{8}{8}\selectfont{(265.3)}}} & \multicolumn{1}{>{\centering}m{\dimexpr 0.7in+0\tabcolsep}}{\textcolor[HTML]{000000}{\fontsize{8}{8}\selectfont{651.0}}\textcolor[HTML]{000000}{\fontsize{8}{8}\selectfont{\linebreak }}\textcolor[HTML]{000000}{\fontsize{8}{8}\selectfont{(27.1)}}} & \multicolumn{1}{>{\centering}m{\dimexpr 0.7in+0\tabcolsep}}{\textcolor[HTML]{000000}{\fontsize{8}{8}\selectfont{1715.2}}\textcolor[HTML]{000000}{\fontsize{8}{8}\selectfont{\linebreak }}\textcolor[HTML]{000000}{\fontsize{8}{8}\selectfont{(240.9)}}} \\





\multicolumn{1}{>{\centering}m{\dimexpr 0.4in+0\tabcolsep}}{\textcolor[HTML]{000000}{\fontsize{8}{8}\selectfont{2,020}}} & \multicolumn{1}{>{\centering}m{\dimexpr 0.5in+0\tabcolsep}}{\textcolor[HTML]{000000}{\fontsize{8}{8}\selectfont{10}}} & \multicolumn{1}{>{\centering}m{\dimexpr 0.4in+0\tabcolsep}}{\textcolor[HTML]{000000}{\fontsize{8}{8}\selectfont{192.5}}\textcolor[HTML]{000000}{\fontsize{8}{8}\selectfont{\linebreak }}\textcolor[HTML]{000000}{\fontsize{8}{8}\selectfont{(42.3)}}} & \multicolumn{1}{>{\centering}m{\dimexpr 0.4in+0\tabcolsep}}{\textcolor[HTML]{000000}{\fontsize{8}{8}\selectfont{33.4}}\textcolor[HTML]{000000}{\fontsize{8}{8}\selectfont{\linebreak }}\textcolor[HTML]{000000}{\fontsize{8}{8}\selectfont{(3.5)}}} & \multicolumn{1}{>{\centering}m{\dimexpr 0.5in+0\tabcolsep}}{\textcolor[HTML]{000000}{\fontsize{8}{8}\selectfont{30.3}}\textcolor[HTML]{000000}{\fontsize{8}{8}\selectfont{\linebreak }}\textcolor[HTML]{000000}{\fontsize{8}{8}\selectfont{(0.2)}}} & \multicolumn{1}{>{\centering}m{\dimexpr 0.7in+0\tabcolsep}}{\textcolor[HTML]{000000}{\fontsize{8}{8}\selectfont{598.4}}\textcolor[HTML]{000000}{\fontsize{8}{8}\selectfont{\linebreak }}\textcolor[HTML]{000000}{\fontsize{8}{8}\selectfont{(145.8)}}} & \multicolumn{1}{>{\centering}m{\dimexpr 0.7in+0\tabcolsep}}{\textcolor[HTML]{000000}{\fontsize{8}{8}\selectfont{1633.7}}\textcolor[HTML]{000000}{\fontsize{8}{8}\selectfont{\linebreak }}\textcolor[HTML]{000000}{\fontsize{8}{8}\selectfont{(154.4)}}} & \multicolumn{1}{>{\centering}m{\dimexpr 0.7in+0\tabcolsep}}{\textcolor[HTML]{000000}{\fontsize{8}{8}\selectfont{626.9}}\textcolor[HTML]{000000}{\fontsize{8}{8}\selectfont{\linebreak }}\textcolor[HTML]{000000}{\fontsize{8}{8}\selectfont{(97.2)}}} & \multicolumn{1}{>{\centering}m{\dimexpr 0.7in+0\tabcolsep}}{\textcolor[HTML]{000000}{\fontsize{8}{8}\selectfont{1649.1}}\textcolor[HTML]{000000}{\fontsize{8}{8}\selectfont{\linebreak }}\textcolor[HTML]{000000}{\fontsize{8}{8}\selectfont{(218.9)}}} \\





\multicolumn{1}{>{\centering}m{\dimexpr 0.4in+0\tabcolsep}}{\textcolor[HTML]{000000}{\fontsize{8}{8}\selectfont{2,021}}} & \multicolumn{1}{>{\centering}m{\dimexpr 0.5in+0\tabcolsep}}{\textcolor[HTML]{000000}{\fontsize{8}{8}\selectfont{14}}} & \multicolumn{1}{>{\centering}m{\dimexpr 0.4in+0\tabcolsep}}{\textcolor[HTML]{000000}{\fontsize{8}{8}\selectfont{198.2}}\textcolor[HTML]{000000}{\fontsize{8}{8}\selectfont{\linebreak }}\textcolor[HTML]{000000}{\fontsize{8}{8}\selectfont{(41.7)}}} & \multicolumn{1}{>{\centering}m{\dimexpr 0.4in+0\tabcolsep}}{\textcolor[HTML]{000000}{\fontsize{8}{8}\selectfont{33.0}}\textcolor[HTML]{000000}{\fontsize{8}{8}\selectfont{\linebreak }}\textcolor[HTML]{000000}{\fontsize{8}{8}\selectfont{(2.7)}}} & \multicolumn{1}{>{\centering}m{\dimexpr 0.5in+0\tabcolsep}}{\textcolor[HTML]{000000}{\fontsize{8}{8}\selectfont{30.4}}\textcolor[HTML]{000000}{\fontsize{8}{8}\selectfont{\linebreak }}\textcolor[HTML]{000000}{\fontsize{8}{8}\selectfont{(2.0)}}} & \multicolumn{1}{>{\centering}m{\dimexpr 0.7in+0\tabcolsep}}{\textcolor[HTML]{000000}{\fontsize{8}{8}\selectfont{618.6}}\textcolor[HTML]{000000}{\fontsize{8}{8}\selectfont{\linebreak }}\textcolor[HTML]{000000}{\fontsize{8}{8}\selectfont{(145.6)}}} & \multicolumn{1}{>{\centering}m{\dimexpr 0.7in+0\tabcolsep}}{\textcolor[HTML]{000000}{\fontsize{8}{8}\selectfont{1811.0}}\textcolor[HTML]{000000}{\fontsize{8}{8}\selectfont{\linebreak }}\textcolor[HTML]{000000}{\fontsize{8}{8}\selectfont{(117.0)}}} & \multicolumn{1}{>{\centering}m{\dimexpr 0.7in+0\tabcolsep}}{\textcolor[HTML]{000000}{\fontsize{8}{8}\selectfont{616.5}}\textcolor[HTML]{000000}{\fontsize{8}{8}\selectfont{\linebreak }}\textcolor[HTML]{000000}{\fontsize{8}{8}\selectfont{(59.0)}}} & \multicolumn{1}{>{\centering}m{\dimexpr 0.7in+0\tabcolsep}}{\textcolor[HTML]{000000}{\fontsize{8}{8}\selectfont{1732.3}}\textcolor[HTML]{000000}{\fontsize{8}{8}\selectfont{\linebreak }}\textcolor[HTML]{000000}{\fontsize{8}{8}\selectfont{(179.0)}}} \\





\multicolumn{1}{>{\centering}m{\dimexpr 0.4in+0\tabcolsep}}{\textcolor[HTML]{000000}{\fontsize{8}{8}\selectfont{2,022}}} & \multicolumn{1}{>{\centering}m{\dimexpr 0.5in+0\tabcolsep}}{\textcolor[HTML]{000000}{\fontsize{8}{8}\selectfont{20}}} & \multicolumn{1}{>{\centering}m{\dimexpr 0.4in+0\tabcolsep}}{\textcolor[HTML]{000000}{\fontsize{8}{8}\selectfont{187.5}}\textcolor[HTML]{000000}{\fontsize{8}{8}\selectfont{\linebreak }}\textcolor[HTML]{000000}{\fontsize{8}{8}\selectfont{(60.1)}}} & \multicolumn{1}{>{\centering}m{\dimexpr 0.4in+0\tabcolsep}}{\textcolor[HTML]{000000}{\fontsize{8}{8}\selectfont{32.4}}\textcolor[HTML]{000000}{\fontsize{8}{8}\selectfont{\linebreak }}\textcolor[HTML]{000000}{\fontsize{8}{8}\selectfont{(3.3)}}} & \multicolumn{1}{>{\centering}m{\dimexpr 0.5in+0\tabcolsep}}{\textcolor[HTML]{000000}{\fontsize{8}{8}\selectfont{29.4}}\textcolor[HTML]{000000}{\fontsize{8}{8}\selectfont{\linebreak }}\textcolor[HTML]{000000}{\fontsize{8}{8}\selectfont{(3.2)}}} & \multicolumn{1}{>{\centering}m{\dimexpr 0.7in+0\tabcolsep}}{\textcolor[HTML]{000000}{\fontsize{8}{8}\selectfont{484.4}}\textcolor[HTML]{000000}{\fontsize{8}{8}\selectfont{\linebreak }}\textcolor[HTML]{000000}{\fontsize{8}{8}\selectfont{(109.6)}}} & \multicolumn{1}{>{\centering}m{\dimexpr 0.7in+0\tabcolsep}}{\textcolor[HTML]{000000}{\fontsize{8}{8}\selectfont{1644.0}}\textcolor[HTML]{000000}{\fontsize{8}{8}\selectfont{\linebreak }}\textcolor[HTML]{000000}{\fontsize{8}{8}\selectfont{(154.6)}}} & \multicolumn{1}{>{\centering}m{\dimexpr 0.7in+0\tabcolsep}}{\textcolor[HTML]{000000}{\fontsize{8}{8}\selectfont{562.9}}\textcolor[HTML]{000000}{\fontsize{8}{8}\selectfont{\linebreak }}\textcolor[HTML]{000000}{\fontsize{8}{8}\selectfont{(94.6)}}} & \multicolumn{1}{>{\centering}m{\dimexpr 0.7in+0\tabcolsep}}{\textcolor[HTML]{000000}{\fontsize{8}{8}\selectfont{1570.0}}\textcolor[HTML]{000000}{\fontsize{8}{8}\selectfont{\linebreak }}\textcolor[HTML]{000000}{\fontsize{8}{8}\selectfont{(164.1)}}} \\





\multicolumn{1}{>{\centering}m{\dimexpr 0.4in+0\tabcolsep}}{\textcolor[HTML]{000000}{\fontsize{8}{8}\selectfont{2,023}}} & \multicolumn{1}{>{\centering}m{\dimexpr 0.5in+0\tabcolsep}}{\textcolor[HTML]{000000}{\fontsize{8}{8}\selectfont{22}}} & \multicolumn{1}{>{\centering}m{\dimexpr 0.4in+0\tabcolsep}}{\textcolor[HTML]{000000}{\fontsize{8}{8}\selectfont{201.9}}\textcolor[HTML]{000000}{\fontsize{8}{8}\selectfont{\linebreak }}\textcolor[HTML]{000000}{\fontsize{8}{8}\selectfont{(50.2)}}} & \multicolumn{1}{>{\centering}m{\dimexpr 0.4in+0\tabcolsep}}{\textcolor[HTML]{000000}{\fontsize{8}{8}\selectfont{32.4}}\textcolor[HTML]{000000}{\fontsize{8}{8}\selectfont{\linebreak }}\textcolor[HTML]{000000}{\fontsize{8}{8}\selectfont{(3.8)}}} & \multicolumn{1}{>{\centering}m{\dimexpr 0.5in+0\tabcolsep}}{\textcolor[HTML]{000000}{\fontsize{8}{8}\selectfont{30.0}}\textcolor[HTML]{000000}{\fontsize{8}{8}\selectfont{\linebreak }}\textcolor[HTML]{000000}{\fontsize{8}{8}\selectfont{(3.0)}}} & \multicolumn{1}{>{\centering}m{\dimexpr 0.7in+0\tabcolsep}}{\textcolor[HTML]{000000}{\fontsize{8}{8}\selectfont{478.5}}\textcolor[HTML]{000000}{\fontsize{8}{8}\selectfont{\linebreak }}\textcolor[HTML]{000000}{\fontsize{8}{8}\selectfont{(49.4)}}} & \multicolumn{1}{>{\centering}m{\dimexpr 0.7in+0\tabcolsep}}{\textcolor[HTML]{000000}{\fontsize{8}{8}\selectfont{1717.0}}\textcolor[HTML]{000000}{\fontsize{8}{8}\selectfont{\linebreak }}\textcolor[HTML]{000000}{\fontsize{8}{8}\selectfont{(166.3)}}} & \multicolumn{1}{>{\centering}m{\dimexpr 0.7in+0\tabcolsep}}{\textcolor[HTML]{000000}{\fontsize{8}{8}\selectfont{603.7}}\textcolor[HTML]{000000}{\fontsize{8}{8}\selectfont{\linebreak }}\textcolor[HTML]{000000}{\fontsize{8}{8}\selectfont{(93.3)}}} & \multicolumn{1}{>{\centering}m{\dimexpr 0.7in+0\tabcolsep}}{\textcolor[HTML]{000000}{\fontsize{8}{8}\selectfont{1641.0}}\textcolor[HTML]{000000}{\fontsize{8}{8}\selectfont{\linebreak }}\textcolor[HTML]{000000}{\fontsize{8}{8}\selectfont{(199.7)}}} \\

\ascline{1.5pt}{666666}{1-9}



\end{longtable*}



\arrayrulecolor[HTML]{000000}

\global\setlength{\arrayrulewidth}{\Oldarrayrulewidth}

\global\setlength{\tabcolsep}{\Oldtabcolsep}

\renewcommand*{\arraystretch}{1}

::: :::

The non-experimental variables of field specific information are added
upon the processed yield and input combined polygon. The public data
resource, Soil survey information(SSURGO) and Digital Eleveation
Model(DEM) are used as a field characteristic information; clay, sand,
silt and water storage for soil, elevation, slope and curvature for
topography. The median value of the soil and topography rasters are
calculated for the overlapped polygon of experimental data by using R
software (Team et al. (\citeproc{ref-r2021language}{2021})). As the
experimental and non-experimental variables are cleared and processed,
for each experimental field, in-season (April 1st to September 30th)
weather information is attached. From the Daily surface weather data,
daymet(Thornton et al. (\citeproc{ref-thornton2022daymet}{2022})),
in-season total precipitation and accumulated growing degree day (GDD)
of the trial year are calculated and added on each of the processed OFPE
data. Finally, annual reported average seeding rate by state (USDASR)
are obtained from NASS (\citeproc{ref-usdanass}{2024}) and attached to
compare the SQSR and USDASR with the estimated profit, respectively.

\textbf{?@tbl-dat\_summary} is the summary statistics of the processed
OFPE data by the trial year (from 2016 t0 2023). Each row of the table
contains the average, standard deviation, minimum and max value of the
experimental data; yield(bu/ac), SQSR(K/ac), USDASR(K/ac) and weather
information; precipitation(inch) and accumulated GDD.

\subsection{Models}\label{models}

\begin{equation}\phantomsection\label{eq-yield}{
Y = f(X,C,Z)
}\end{equation}

Equation~\ref{eq-yield} is a meta yield response function which
describes the true crop yield response function with respect to
input(Bullock et al. (\citeproc{ref-bullock2009value}{2009})). Corn
yield \(\mathbf{Y}\) is a function of controllable inputs
\(\mathbf{X}\), field characteristics \(\mathbf{C}\), and climate
condition \(\mathbf{Z}\). S and N fertilizer are the most important
manageable inputs which belongs to \(\mathbf{X}\) since their share of
total farm operation costs are higest among all the inputs.
\(\mathbf{C}\) is consist of soil and topographic features, which are
spatially depdendent and not changeable or very slowly varient during
the single or couple of crop cultivating seasons. \(\mathbf{Z}\) is
stocahastic and highly time-variable factor that is not observable when
the farmer make a decision of their inputs, \(\mathbf{X}\).

\begin{equation}\phantomsection\label{eq-res-ante}{
E(Y_{it} \mid z_t) =  f_{it}(X,C_{it})
}\end{equation}

Equation~\ref{eq-res-ante} represents the ex-ante yield response
function at a given experimental field \(i\) in a year \(t\). The
expected yield \(E(Y_{it})\), at the moment of input rates decision,
depends on the stochastic weather event during the in-season of year
\(t\), \(\mathbf{z_t} \in Z\), and its associated probability. \(z_t\)
is not observable in the decision timing, also it is hard to forecast,
so farmers need to make strategic input decision first, based on their
knowledge about the functional form of \(f_{it}\), and the given market
price of corn and the input price, finally, with the expectation about
the future weather scenario of \(z_t\).

Once the OFPE has been executed in the field \(i\), and all the required
data are collected, the ex-post yield input response function can be
estimated with the observed weather \(z_t\). Then, the estimated ex-post
yield response function becomes

\begin{equation}\phantomsection\label{eq-res-post}{
Y_{it} = \hat{f}_{it}(X_{it}, \hat{C}_{it}) + \epsilon_{it}
}\end{equation}

In Equation~\ref{eq-res-post}, by acquiring the incorporated OFPE data,
soil and topographic feature of the field, \(\hat{C}_{it}\) is observed,
and the estimated yield response function,\(\hat{f}_{it}\) can be
calculated.

Farmer's main purpose of input decision, is to maximize the expected
profit with their own knowledge about yield response function,
\(\theta\). Equation~\ref{eq-res-ante-farmer} represents the expected
maximum of farmer's profit where \(\mathit{h}(\mathbf{z})\) is the joint
probability density function of weather events \(\mathbf{z} \in Z\).
\begin{equation}\phantomsection\label{eq-res-ante-farmer}{
E(\pi_i|\theta) \equiv \max_{X_{ij}} \ \mathbf{p} \int_{z \in Z} \left\{f_{it}^\theta(X_{ij}) - W \cdot X_{ij} \right\} h(\mathrm{z})dz
}\end{equation}

\(\theta\) is established upon farmer's experiences and infromation, and
it incorporates the farmer's understanding about how manageable inputs
\(X_{ij}\) and field characters \(C_{it}\) interacts each other to form
a farmer's own subjective belief about \(f_{it}^\theta\).

The objective of this research is to evaluate the farmer's seeding rate
decision by cross-validate the profit of farmer's chosen rate and
estimated EOSR of the given experimental field \(i\). To evaluate the
profit of farmer's SQSR and EOSR from the OPFE data, first, field
specific yield input response function is to be accurately estimated. In
estimating yield S response, defining a unique functional form can lead
to severe bias since the function is highly variable by the distribution
of field characteristics and the occured weather event. Thus, this study
adopts Generalized Additive Model(GAM) to estimate yield S response
function. GAM has an advantage when the yield response function has a
various form by field since it arbitrary chooses functional form without
making any assumption on input and output relation. To simplify the
analysis and focus on the S rate management, this study consider S as an
only controllable input.

\begin{equation}\phantomsection\label{eq-res-gam}{
Y_{i} = \beta_0 + g_{1i}(S_{i},k) + g_{2i}(N_{i},k) + \sum_{m=1}^{M} h_{i}(C_{im},k) + \epsilon_i
}\end{equation}

Equation~\ref{eq-res-gam} describes the functional form of GAM
regression model. \(g_{1i}\) and \(g_{2i}\) is the corresponding spline
function of S and N, respectively. k is the number of knots which
determines the flexibility of the spline function used to model the
relationship of yield and response input (Wood
(\citeproc{ref-wood2017generalized}{2017})). Once the \(k=0\), the
spline function becomes linear, and for the \(k>0\), the higher \(k\)
indicates more flexibility in spline curve to fit into the data points.
For the \(g_{1i}\) and \(g_{2i}\), the knot is resticed to be,
\(k\in(0,3,4)\). By a number of test results with the colleceted OFPE
data, the higher \(k>4\) frequently bring out the over-fitting problem
by generating wiggly response curve, meanwhile the fewer knots
\(k\in(1,2)\) led to underfitting problem by lowering the generalized
cross-validation(GCV) score drastically. Spline function of all the
\(m\) number of field characteristic variables are indentical and
denoted as, \(h_i\) to avoid complexity of the model which lower the GCV
scores.

In this research, the estimtaded yield response function from
Equation~\ref{eq-res-gam} is assumed to be true yield response since it
becomes teh criteria of the profit evaluation. By plugging this
estitmates into the farmer's profit maximization in
Equation~\ref{eq-res-ante-farmer}, ex-post esmiateed profit maximization
becomes,

\begin{equation}\phantomsection\label{eq-res-post-farmer}{
\pi_i(S{_i}^*(\bar{p}, \bar{w}))  \equiv \max_{S_{i}} \ \mathbf{\bar{p}} \left(\beta_0 + g_{1i}(S_{i},k) + g_{2i}(N_{i},k) + \sum_{m=1}^{M} h_{i}(C_{im},k)\right) - \bar{w} S_i
}\end{equation}

Equation Equation~\ref{eq-res-post-farmer} is the estimated per acre net
revenue after seeding cost under the given market price of input
\(\bar{p}\) and output \(\bar{w}\) .
\(\left(S{_i}^*(\bar{p}, \bar{w})\right)\) is the uniform EOSR of the
field \(i\), which maximize the the profit of the field,\(\pi{_i}\).
Farmer's own choice of seeding rate, SQSR, is denoted as
\(\left(S{_i}^{sq}(\bar{p}, \bar{w})\right)\).

\begin{equation}\phantomsection\label{eq-prof_dif}{
   E[Pr{_i}]  \equiv  \pi{_i} \left(S{_i}^{*}(\bar{p}, \bar{w})\right) -  \pi{_i} \left(S{_i}^{sq}(\bar{p}, \bar{w})\right)
}\end{equation}

In Equation~\ref{eq-prof_dif}, \(E[Pr{_i}]\) is the potential expected
profit per acre that farmer can acquire by choosing EOSR instead of
SQSR. As the farmers have better information about yield response of the
field \(i\), farmer would make their seeding decision,
\(\left(S{_i}^{sq}(\bar{p} \bar{w})\right)\) to be closer with
\(\left(S{_i}^*(\bar{p}, \bar{w})\right)\).

\section{Results}\label{results}

\begin{figure}

\centering{

\includegraphics[width=1\textwidth,height=\textheight]{corn_seed_response_writing_files/figure-pdf/fig-eosr-sqsr-weather-1.pdf}

}

\caption{\label{fig-eosr-sqsr-weather}EOSR and SQSR by the in-season
precipitation (left) \& 30 year Avg Precipitation (right)}

\end{figure}%

Figure Figure~\ref{fig-eosr-sqsr-weather} illustrates the relationship
between climate conditions and seeding rates. The left panel compares
EOSRs (red) and SQSRs (blue) based on in-season total precipitation
(inches) across 100 field trials, while the right panel shows the
comparison based on accumulated Growing Degree Days (GDD). EOSRs exhibit
greater vertical dispersion under the same levels of precipitation or
GDD, whereas SQSRs show less variation relative to EOSRs. In 93 out of
the 100 trials, SQSRs range from 30K to 36K, with a median value of 34K.
In contrast, the median EOSR is 30.2K.

Additionally, Figure Figure~\ref{fig-eosr-sqsr-weather} reveals that,
across the 100 trials, farmers' seeding rates tend to be higher than the
estimated optimal, and this difference is skewed to the left when
projected onto the diagonal line (black), indicating a tendency for
over-seeding relative to the EOSR.

\global\setlength{\Oldarrayrulewidth}{\arrayrulewidth}

\global\setlength{\Oldtabcolsep}{\tabcolsep}

\setlength{\tabcolsep}{2pt}

\renewcommand*{\arraystretch}{1.5}



\providecommand{\ascline}[3]{\noalign{\global\arrayrulewidth #1}\arrayrulecolor[HTML]{#2}\cline{#3}}

\begin{longtable*}[c]{|p{0.50in}|p{0.40in}|p{0.60in}|p{0.60in}|p{0.60in}|p{0.60in}|p{0.60in}|p{0.60in}|p{0.60in}}



\ascline{1.5pt}{666666}{1-9}

\multicolumn{1}{>{\centering}m{\dimexpr 0.5in+0\tabcolsep}}{\textcolor[HTML]{000000}{\fontsize{8}{8}\selectfont{Response}}\textcolor[HTML]{000000}{\fontsize{8}{8}\selectfont{\linebreak }}\textcolor[HTML]{000000}{\fontsize{8}{8}\selectfont{Type}}} & \multicolumn{1}{>{\centering}m{\dimexpr 0.4in+0\tabcolsep}}{\textcolor[HTML]{000000}{\fontsize{8}{8}\selectfont{Field}}\textcolor[HTML]{000000}{\fontsize{8}{8}\selectfont{\linebreak }}\textcolor[HTML]{000000}{\fontsize{8}{8}\selectfont{Count}}} & \multicolumn{1}{>{\centering}m{\dimexpr 0.6in+0\tabcolsep}}{\textcolor[HTML]{000000}{\fontsize{8}{8}\selectfont{Differences}}\textcolor[HTML]{000000}{\fontsize{8}{8}\selectfont{\linebreak }}\textcolor[HTML]{000000}{\fontsize{8}{8}\selectfont{in}}\textcolor[HTML]{000000}{\fontsize{8}{8}\selectfont{\linebreak }}\textcolor[HTML]{000000}{\fontsize{8}{8}\selectfont{\ Seeding\ Rate}}\textcolor[HTML]{000000}{\fontsize{8}{8}\selectfont{\linebreak }}\textcolor[HTML]{000000}{\fontsize{8}{8}\selectfont{(K/ac)}}} & \multicolumn{1}{>{\centering}m{\dimexpr 0.6in+0\tabcolsep}}{\textcolor[HTML]{000000}{\fontsize{8}{8}\selectfont{Differences}}\textcolor[HTML]{000000}{\fontsize{8}{8}\selectfont{\linebreak }}\textcolor[HTML]{000000}{\fontsize{8}{8}\selectfont{in}}\textcolor[HTML]{000000}{\fontsize{8}{8}\selectfont{\linebreak }}\textcolor[HTML]{000000}{\fontsize{8}{8}\selectfont{\ Estimated\ Yield}}\textcolor[HTML]{000000}{\fontsize{8}{8}\selectfont{\linebreak }}\textcolor[HTML]{000000}{\fontsize{8}{8}\selectfont{(bu/ac)}}} & \multicolumn{1}{>{\centering}m{\dimexpr 0.6in+0\tabcolsep}}{\textcolor[HTML]{000000}{\fontsize{8}{8}\selectfont{Differences}}\textcolor[HTML]{000000}{\fontsize{8}{8}\selectfont{\linebreak }}\textcolor[HTML]{000000}{\fontsize{8}{8}\selectfont{in}}\textcolor[HTML]{000000}{\fontsize{8}{8}\selectfont{\linebreak }}\textcolor[HTML]{000000}{\fontsize{8}{8}\selectfont{\ Estimated\ Profit}}\textcolor[HTML]{000000}{\fontsize{8}{8}\selectfont{\linebreak }}\textcolor[HTML]{000000}{\fontsize{8}{8}\selectfont{(\$/ac)}}} & \multicolumn{1}{>{\centering}m{\dimexpr 0.6in+0\tabcolsep}}{\textcolor[HTML]{000000}{\fontsize{8}{8}\selectfont{Precipitation}}\textcolor[HTML]{000000}{\fontsize{8}{8}\selectfont{\linebreak }}\textcolor[HTML]{000000}{\fontsize{8}{8}\selectfont{(In-Season)}}} & \multicolumn{1}{>{\centering}m{\dimexpr 0.6in+0\tabcolsep}}{\textcolor[HTML]{000000}{\fontsize{8}{8}\selectfont{GDD}}\textcolor[HTML]{000000}{\fontsize{8}{8}\selectfont{\linebreak }}\textcolor[HTML]{000000}{\fontsize{8}{8}\selectfont{(In-Season)}}} & \multicolumn{1}{>{\centering}m{\dimexpr 0.6in+0\tabcolsep}}{\textcolor[HTML]{000000}{\fontsize{8}{8}\selectfont{Precipitation}}\textcolor[HTML]{000000}{\fontsize{8}{8}\selectfont{\linebreak }}\textcolor[HTML]{000000}{\fontsize{8}{8}\selectfont{(30\ Year)}}} & \multicolumn{1}{>{\centering}m{\dimexpr 0.6in+0\tabcolsep}}{\textcolor[HTML]{000000}{\fontsize{8}{8}\selectfont{GDD}}\textcolor[HTML]{000000}{\fontsize{8}{8}\selectfont{\linebreak }}\textcolor[HTML]{000000}{\fontsize{8}{8}\selectfont{(30\ Year)}}} \\

\ascline{1.5pt}{666666}{1-9}\endfirsthead 

\ascline{1.5pt}{666666}{1-9}

\multicolumn{1}{>{\centering}m{\dimexpr 0.5in+0\tabcolsep}}{\textcolor[HTML]{000000}{\fontsize{8}{8}\selectfont{Response}}\textcolor[HTML]{000000}{\fontsize{8}{8}\selectfont{\linebreak }}\textcolor[HTML]{000000}{\fontsize{8}{8}\selectfont{Type}}} & \multicolumn{1}{>{\centering}m{\dimexpr 0.4in+0\tabcolsep}}{\textcolor[HTML]{000000}{\fontsize{8}{8}\selectfont{Field}}\textcolor[HTML]{000000}{\fontsize{8}{8}\selectfont{\linebreak }}\textcolor[HTML]{000000}{\fontsize{8}{8}\selectfont{Count}}} & \multicolumn{1}{>{\centering}m{\dimexpr 0.6in+0\tabcolsep}}{\textcolor[HTML]{000000}{\fontsize{8}{8}\selectfont{Differences}}\textcolor[HTML]{000000}{\fontsize{8}{8}\selectfont{\linebreak }}\textcolor[HTML]{000000}{\fontsize{8}{8}\selectfont{in}}\textcolor[HTML]{000000}{\fontsize{8}{8}\selectfont{\linebreak }}\textcolor[HTML]{000000}{\fontsize{8}{8}\selectfont{\ Seeding\ Rate}}\textcolor[HTML]{000000}{\fontsize{8}{8}\selectfont{\linebreak }}\textcolor[HTML]{000000}{\fontsize{8}{8}\selectfont{(K/ac)}}} & \multicolumn{1}{>{\centering}m{\dimexpr 0.6in+0\tabcolsep}}{\textcolor[HTML]{000000}{\fontsize{8}{8}\selectfont{Differences}}\textcolor[HTML]{000000}{\fontsize{8}{8}\selectfont{\linebreak }}\textcolor[HTML]{000000}{\fontsize{8}{8}\selectfont{in}}\textcolor[HTML]{000000}{\fontsize{8}{8}\selectfont{\linebreak }}\textcolor[HTML]{000000}{\fontsize{8}{8}\selectfont{\ Estimated\ Yield}}\textcolor[HTML]{000000}{\fontsize{8}{8}\selectfont{\linebreak }}\textcolor[HTML]{000000}{\fontsize{8}{8}\selectfont{(bu/ac)}}} & \multicolumn{1}{>{\centering}m{\dimexpr 0.6in+0\tabcolsep}}{\textcolor[HTML]{000000}{\fontsize{8}{8}\selectfont{Differences}}\textcolor[HTML]{000000}{\fontsize{8}{8}\selectfont{\linebreak }}\textcolor[HTML]{000000}{\fontsize{8}{8}\selectfont{in}}\textcolor[HTML]{000000}{\fontsize{8}{8}\selectfont{\linebreak }}\textcolor[HTML]{000000}{\fontsize{8}{8}\selectfont{\ Estimated\ Profit}}\textcolor[HTML]{000000}{\fontsize{8}{8}\selectfont{\linebreak }}\textcolor[HTML]{000000}{\fontsize{8}{8}\selectfont{(\$/ac)}}} & \multicolumn{1}{>{\centering}m{\dimexpr 0.6in+0\tabcolsep}}{\textcolor[HTML]{000000}{\fontsize{8}{8}\selectfont{Precipitation}}\textcolor[HTML]{000000}{\fontsize{8}{8}\selectfont{\linebreak }}\textcolor[HTML]{000000}{\fontsize{8}{8}\selectfont{(In-Season)}}} & \multicolumn{1}{>{\centering}m{\dimexpr 0.6in+0\tabcolsep}}{\textcolor[HTML]{000000}{\fontsize{8}{8}\selectfont{GDD}}\textcolor[HTML]{000000}{\fontsize{8}{8}\selectfont{\linebreak }}\textcolor[HTML]{000000}{\fontsize{8}{8}\selectfont{(In-Season)}}} & \multicolumn{1}{>{\centering}m{\dimexpr 0.6in+0\tabcolsep}}{\textcolor[HTML]{000000}{\fontsize{8}{8}\selectfont{Precipitation}}\textcolor[HTML]{000000}{\fontsize{8}{8}\selectfont{\linebreak }}\textcolor[HTML]{000000}{\fontsize{8}{8}\selectfont{(30\ Year)}}} & \multicolumn{1}{>{\centering}m{\dimexpr 0.6in+0\tabcolsep}}{\textcolor[HTML]{000000}{\fontsize{8}{8}\selectfont{GDD}}\textcolor[HTML]{000000}{\fontsize{8}{8}\selectfont{\linebreak }}\textcolor[HTML]{000000}{\fontsize{8}{8}\selectfont{(30\ Year)}}} \\

\ascline{1.5pt}{666666}{1-9}\endhead



\multicolumn{1}{>{\centering}m{\dimexpr 0.5in+0\tabcolsep}}{\textcolor[HTML]{000000}{\fontsize{8}{8}\selectfont{A1}}} & \multicolumn{1}{>{\centering}m{\dimexpr 0.4in+0\tabcolsep}}{\textcolor[HTML]{000000}{\fontsize{8}{8}\selectfont{12}}} & \multicolumn{1}{>{\centering}m{\dimexpr 0.6in+0\tabcolsep}}{\textcolor[HTML]{000000}{\fontsize{8}{8}\selectfont{-4.4}}\textcolor[HTML]{000000}{\fontsize{8}{8}\selectfont{\linebreak }}\textcolor[HTML]{000000}{\fontsize{8}{8}\selectfont{(2.2)}}} & \multicolumn{1}{>{\centering}m{\dimexpr 0.6in+0\tabcolsep}}{\textcolor[HTML]{000000}{\fontsize{8}{8}\selectfont{-5.1}}\textcolor[HTML]{000000}{\fontsize{8}{8}\selectfont{\linebreak }}\textcolor[HTML]{000000}{\fontsize{8}{8}\selectfont{(2.8)}}} & \multicolumn{1}{>{\centering}m{\dimexpr 0.6in+0\tabcolsep}}{\textcolor[HTML]{000000}{\fontsize{8}{8}\selectfont{-13.4}}\textcolor[HTML]{000000}{\fontsize{8}{8}\selectfont{\linebreak }}\textcolor[HTML]{000000}{\fontsize{8}{8}\selectfont{(12.3)}}} & \multicolumn{1}{>{\centering}m{\dimexpr 0.6in+0\tabcolsep}}{\textcolor[HTML]{000000}{\fontsize{8}{8}\selectfont{611.6}}\textcolor[HTML]{000000}{\fontsize{8}{8}\selectfont{\linebreak }}\textcolor[HTML]{000000}{\fontsize{8}{8}\selectfont{(140.0)}}} & \multicolumn{1}{>{\centering}m{\dimexpr 0.6in+0\tabcolsep}}{\textcolor[HTML]{000000}{\fontsize{8}{8}\selectfont{1781.7}}\textcolor[HTML]{000000}{\fontsize{8}{8}\selectfont{\linebreak }}\textcolor[HTML]{000000}{\fontsize{8}{8}\selectfont{(224.0)}}} & \multicolumn{1}{>{\centering}m{\dimexpr 0.6in+0\tabcolsep}}{\textcolor[HTML]{000000}{\fontsize{8}{8}\selectfont{596.1}}\textcolor[HTML]{000000}{\fontsize{8}{8}\selectfont{\linebreak }}\textcolor[HTML]{000000}{\fontsize{8}{8}\selectfont{(80.9)}}} & \multicolumn{1}{>{\centering}m{\dimexpr 0.6in+0\tabcolsep}}{\textcolor[HTML]{000000}{\fontsize{8}{8}\selectfont{1687.9}}\textcolor[HTML]{000000}{\fontsize{8}{8}\selectfont{\linebreak }}\textcolor[HTML]{000000}{\fontsize{8}{8}\selectfont{(206.7)}}} \\





\multicolumn{1}{>{\centering}m{\dimexpr 0.5in+0\tabcolsep}}{\textcolor[HTML]{000000}{\fontsize{8}{8}\selectfont{A2}}} & \multicolumn{1}{>{\centering}m{\dimexpr 0.4in+0\tabcolsep}}{\textcolor[HTML]{000000}{\fontsize{8}{8}\selectfont{12}}} & \multicolumn{1}{>{\centering}m{\dimexpr 0.6in+0\tabcolsep}}{\textcolor[HTML]{000000}{\fontsize{8}{8}\selectfont{-2.2}}\textcolor[HTML]{000000}{\fontsize{8}{8}\selectfont{\linebreak }}\textcolor[HTML]{000000}{\fontsize{8}{8}\selectfont{(1.7)}}} & \multicolumn{1}{>{\centering}m{\dimexpr 0.6in+0\tabcolsep}}{\textcolor[HTML]{000000}{\fontsize{8}{8}\selectfont{-2.4}}\textcolor[HTML]{000000}{\fontsize{8}{8}\selectfont{\linebreak }}\textcolor[HTML]{000000}{\fontsize{8}{8}\selectfont{(2.2)}}} & \multicolumn{1}{>{\centering}m{\dimexpr 0.6in+0\tabcolsep}}{\textcolor[HTML]{000000}{\fontsize{8}{8}\selectfont{-5.3}}\textcolor[HTML]{000000}{\fontsize{8}{8}\selectfont{\linebreak }}\textcolor[HTML]{000000}{\fontsize{8}{8}\selectfont{(6.7)}}} & \multicolumn{1}{>{\centering}m{\dimexpr 0.6in+0\tabcolsep}}{\textcolor[HTML]{000000}{\fontsize{8}{8}\selectfont{562.6}}\textcolor[HTML]{000000}{\fontsize{8}{8}\selectfont{\linebreak }}\textcolor[HTML]{000000}{\fontsize{8}{8}\selectfont{(160.7)}}} & \multicolumn{1}{>{\centering}m{\dimexpr 0.6in+0\tabcolsep}}{\textcolor[HTML]{000000}{\fontsize{8}{8}\selectfont{1779.5}}\textcolor[HTML]{000000}{\fontsize{8}{8}\selectfont{\linebreak }}\textcolor[HTML]{000000}{\fontsize{8}{8}\selectfont{(216.7)}}} & \multicolumn{1}{>{\centering}m{\dimexpr 0.6in+0\tabcolsep}}{\textcolor[HTML]{000000}{\fontsize{8}{8}\selectfont{622.6}}\textcolor[HTML]{000000}{\fontsize{8}{8}\selectfont{\linebreak }}\textcolor[HTML]{000000}{\fontsize{8}{8}\selectfont{(84.8)}}} & \multicolumn{1}{>{\centering}m{\dimexpr 0.6in+0\tabcolsep}}{\textcolor[HTML]{000000}{\fontsize{8}{8}\selectfont{1714.8}}\textcolor[HTML]{000000}{\fontsize{8}{8}\selectfont{\linebreak }}\textcolor[HTML]{000000}{\fontsize{8}{8}\selectfont{(242.8)}}} \\





\multicolumn{1}{>{\centering}m{\dimexpr 0.5in+0\tabcolsep}}{\textcolor[HTML]{000000}{\fontsize{8}{8}\selectfont{B1}}} & \multicolumn{1}{>{\centering}m{\dimexpr 0.4in+0\tabcolsep}}{\textcolor[HTML]{000000}{\fontsize{8}{8}\selectfont{12}}} & \multicolumn{1}{>{\centering}m{\dimexpr 0.6in+0\tabcolsep}}{\textcolor[HTML]{000000}{\fontsize{8}{8}\selectfont{9.4}}\textcolor[HTML]{000000}{\fontsize{8}{8}\selectfont{\linebreak }}\textcolor[HTML]{000000}{\fontsize{8}{8}\selectfont{(3.4)}}} & \multicolumn{1}{>{\centering}m{\dimexpr 0.6in+0\tabcolsep}}{\textcolor[HTML]{000000}{\fontsize{8}{8}\selectfont{-8.2}}\textcolor[HTML]{000000}{\fontsize{8}{8}\selectfont{\linebreak }}\textcolor[HTML]{000000}{\fontsize{8}{8}\selectfont{(10.8)}}} & \multicolumn{1}{>{\centering}m{\dimexpr 0.6in+0\tabcolsep}}{\textcolor[HTML]{000000}{\fontsize{8}{8}\selectfont{-74.5}}\textcolor[HTML]{000000}{\fontsize{8}{8}\selectfont{\linebreak }}\textcolor[HTML]{000000}{\fontsize{8}{8}\selectfont{(61.5)}}} & \multicolumn{1}{>{\centering}m{\dimexpr 0.6in+0\tabcolsep}}{\textcolor[HTML]{000000}{\fontsize{8}{8}\selectfont{511.7}}\textcolor[HTML]{000000}{\fontsize{8}{8}\selectfont{\linebreak }}\textcolor[HTML]{000000}{\fontsize{8}{8}\selectfont{(190.3)}}} & \multicolumn{1}{>{\centering}m{\dimexpr 0.6in+0\tabcolsep}}{\textcolor[HTML]{000000}{\fontsize{8}{8}\selectfont{1678.5}}\textcolor[HTML]{000000}{\fontsize{8}{8}\selectfont{\linebreak }}\textcolor[HTML]{000000}{\fontsize{8}{8}\selectfont{(167.7)}}} & \multicolumn{1}{>{\centering}m{\dimexpr 0.6in+0\tabcolsep}}{\textcolor[HTML]{000000}{\fontsize{8}{8}\selectfont{557.8}}\textcolor[HTML]{000000}{\fontsize{8}{8}\selectfont{\linebreak }}\textcolor[HTML]{000000}{\fontsize{8}{8}\selectfont{(93.7)}}} & \multicolumn{1}{>{\centering}m{\dimexpr 0.6in+0\tabcolsep}}{\textcolor[HTML]{000000}{\fontsize{8}{8}\selectfont{1563.7}}\textcolor[HTML]{000000}{\fontsize{8}{8}\selectfont{\linebreak }}\textcolor[HTML]{000000}{\fontsize{8}{8}\selectfont{(175.7)}}} \\





\multicolumn{1}{>{\centering}m{\dimexpr 0.5in+0\tabcolsep}}{\textcolor[HTML]{000000}{\fontsize{8}{8}\selectfont{B2}}} & \multicolumn{1}{>{\centering}m{\dimexpr 0.4in+0\tabcolsep}}{\textcolor[HTML]{000000}{\fontsize{8}{8}\selectfont{42}}} & \multicolumn{1}{>{\centering}m{\dimexpr 0.6in+0\tabcolsep}}{\textcolor[HTML]{000000}{\fontsize{8}{8}\selectfont{5.3}}\textcolor[HTML]{000000}{\fontsize{8}{8}\selectfont{\linebreak }}\textcolor[HTML]{000000}{\fontsize{8}{8}\selectfont{(4.0)}}} & \multicolumn{1}{>{\centering}m{\dimexpr 0.6in+0\tabcolsep}}{\textcolor[HTML]{000000}{\fontsize{8}{8}\selectfont{-1.7}}\textcolor[HTML]{000000}{\fontsize{8}{8}\selectfont{\linebreak }}\textcolor[HTML]{000000}{\fontsize{8}{8}\selectfont{(6.5)}}} & \multicolumn{1}{>{\centering}m{\dimexpr 0.6in+0\tabcolsep}}{\textcolor[HTML]{000000}{\fontsize{8}{8}\selectfont{-26.3}}\textcolor[HTML]{000000}{\fontsize{8}{8}\selectfont{\linebreak }}\textcolor[HTML]{000000}{\fontsize{8}{8}\selectfont{(41.9)}}} & \multicolumn{1}{>{\centering}m{\dimexpr 0.6in+0\tabcolsep}}{\textcolor[HTML]{000000}{\fontsize{8}{8}\selectfont{570.5}}\textcolor[HTML]{000000}{\fontsize{8}{8}\selectfont{\linebreak }}\textcolor[HTML]{000000}{\fontsize{8}{8}\selectfont{(115.4)}}} & \multicolumn{1}{>{\centering}m{\dimexpr 0.6in+0\tabcolsep}}{\textcolor[HTML]{000000}{\fontsize{8}{8}\selectfont{1747.5}}\textcolor[HTML]{000000}{\fontsize{8}{8}\selectfont{\linebreak }}\textcolor[HTML]{000000}{\fontsize{8}{8}\selectfont{(172.8)}}} & \multicolumn{1}{>{\centering}m{\dimexpr 0.6in+0\tabcolsep}}{\textcolor[HTML]{000000}{\fontsize{8}{8}\selectfont{620.8}}\textcolor[HTML]{000000}{\fontsize{8}{8}\selectfont{\linebreak }}\textcolor[HTML]{000000}{\fontsize{8}{8}\selectfont{(70.8)}}} & \multicolumn{1}{>{\centering}m{\dimexpr 0.6in+0\tabcolsep}}{\textcolor[HTML]{000000}{\fontsize{8}{8}\selectfont{1677.0}}\textcolor[HTML]{000000}{\fontsize{8}{8}\selectfont{\linebreak }}\textcolor[HTML]{000000}{\fontsize{8}{8}\selectfont{(185.5)}}} \\





\multicolumn{1}{>{\centering}m{\dimexpr 0.5in+0\tabcolsep}}{\textcolor[HTML]{000000}{\fontsize{8}{8}\selectfont{C}}} & \multicolumn{1}{>{\centering}m{\dimexpr 0.4in+0\tabcolsep}}{\textcolor[HTML]{000000}{\fontsize{8}{8}\selectfont{18}}} & \multicolumn{1}{>{\centering}m{\dimexpr 0.6in+0\tabcolsep}}{\textcolor[HTML]{000000}{\fontsize{8}{8}\selectfont{6.4}}\textcolor[HTML]{000000}{\fontsize{8}{8}\selectfont{\linebreak }}\textcolor[HTML]{000000}{\fontsize{8}{8}\selectfont{(4.5)}}} & \multicolumn{1}{>{\centering}m{\dimexpr 0.6in+0\tabcolsep}}{\textcolor[HTML]{000000}{\fontsize{8}{8}\selectfont{-6.4}}\textcolor[HTML]{000000}{\fontsize{8}{8}\selectfont{\linebreak }}\textcolor[HTML]{000000}{\fontsize{8}{8}\selectfont{(9.7)}}} & \multicolumn{1}{>{\centering}m{\dimexpr 0.6in+0\tabcolsep}}{\textcolor[HTML]{000000}{\fontsize{8}{8}\selectfont{-50.6}}\textcolor[HTML]{000000}{\fontsize{8}{8}\selectfont{\linebreak }}\textcolor[HTML]{000000}{\fontsize{8}{8}\selectfont{(60.5)}}} & \multicolumn{1}{>{\centering}m{\dimexpr 0.6in+0\tabcolsep}}{\textcolor[HTML]{000000}{\fontsize{8}{8}\selectfont{643.3}}\textcolor[HTML]{000000}{\fontsize{8}{8}\selectfont{\linebreak }}\textcolor[HTML]{000000}{\fontsize{8}{8}\selectfont{(149.5)}}} & \multicolumn{1}{>{\centering}m{\dimexpr 0.6in+0\tabcolsep}}{\textcolor[HTML]{000000}{\fontsize{8}{8}\selectfont{1799.5}}\textcolor[HTML]{000000}{\fontsize{8}{8}\selectfont{\linebreak }}\textcolor[HTML]{000000}{\fontsize{8}{8}\selectfont{(208.6)}}} & \multicolumn{1}{>{\centering}m{\dimexpr 0.6in+0\tabcolsep}}{\textcolor[HTML]{000000}{\fontsize{8}{8}\selectfont{621.9}}\textcolor[HTML]{000000}{\fontsize{8}{8}\selectfont{\linebreak }}\textcolor[HTML]{000000}{\fontsize{8}{8}\selectfont{(69.1)}}} & \multicolumn{1}{>{\centering}m{\dimexpr 0.6in+0\tabcolsep}}{\textcolor[HTML]{000000}{\fontsize{8}{8}\selectfont{1709.0}}\textcolor[HTML]{000000}{\fontsize{8}{8}\selectfont{\linebreak }}\textcolor[HTML]{000000}{\fontsize{8}{8}\selectfont{(189.6)}}} \\

\ascline{1.5pt}{666666}{1-9}



\end{longtable*}



\arrayrulecolor[HTML]{000000}

\global\setlength{\arrayrulewidth}{\Oldarrayrulewidth}

\global\setlength{\tabcolsep}{\Oldtabcolsep}

\renewcommand*{\arraystretch}{1}

For a more detailed evaluation of farmers' seeding rate decisions, table
(\citeproc{ref-tab-result_resp_type}{\textbf{tab-result\_resp\_type?}})
presents the differences in seeding rate, yield, and profit at the SQSR
and EOSR, with the 100 trials divided into five categories. These
categories are based on the concavity of the yield response function,
the sign of the difference between SQSR and EOSR, and whether the EOSR
matches the Yield Maximizing Seeding Rate (YMSR).

The concavity of the yield response function is a crucial factor in
economic analysis, as a convex response typically results in a corner
solution at either the minimum or maximum value of the input, providing
little useful guidance for determining an optimal seeding rate. In
contrast, when the EOSR equals the YMSR, in some cases, the EOSR may
also occur at the minimum or maximum of the trial input range. In these
instances, the EOSR offers limited insight, as it only applies within
the specific range of inputs used in the trial.

In table
(\citeproc{ref-tab-result_resp_type}{\textbf{tab-result\_resp\_type?}}),
the overall difference between the SQSRs and EOSRs averages -3.8K,
meaning that, on average, farmers are seeding 3.8K more seeds per acre
than the estimated optimal rate. The estimated yield at SQSRs is 3.4
bushels per acre less than at EOSRs, reflecting the diminishing yield
effect when seeding rates exceed the EOSR. Types A1 and B1 represent
trials where the yield response function is concave, but the EOSR is
identical to the YMSR. In many A1 type trials, the EOSRs are at or near
the maximum of the trial input range, while in B1 type trials, the EOSRs
are at or near the minimum of the trial input range. The average seeding
rate differences are -4.5K in A1 and 9.1K in B1, indicating that, in the
24 trials classified as A1 and B1, farmers tend to choose higher seeding
rates than the EOSR. The average yield difference between A1 and B1 is
3.1 bushels per acre, but the estimated profit difference between these
groups is \$69.3 per acre. This result highlights that the potential
profit loss from farmers' SQSR choices is largely due to the cost of
excessive seeding. However, because some trials have EOSRs at the
extremes of the input range (either the maximum or minimum), the numeric
results for groups A1 and B1 may be biased.

Types A2 and B2 represent trials where the EOSRs are the most accurate
and informative, as the yield response functions are concave, and the
EOSRs are neither at the maximum nor the minimum of the trial input
ranges. In these groups, the average yield difference between the two is
only 1.5 bushels per acre, but the estimated profit difference is \$18.8
per acre. Notably, the B2 group comprises 41 out of the 100 total
trials, indicating that farmers tend to plant excessive seeds in many
cases. This over-seeding leads to significant profit losses due to
unproductive investment in seed.

\begin{figure}

\centering{

\includegraphics[width=1\textwidth,height=\textheight]{corn_seed_response_writing_files/figure-pdf/fig-dif_pro_seed_comb-1.pdf}

}

\caption{\label{fig-dif_pro_seed_comb}Difference in estimated Profit at
a given climate condition by seeding status (SQSR-EOSR)}

\end{figure}%

Figure~\ref{fig-dif_pro_seed_comb} displays the differences in Estimated
Profit and Seeding Rate between the SQSR and EOSR , with the top panel
showing these differences as a function of total in-sesaon precipitation
of the trial year and the bottom panel categorizing them by the 30 year
average precipitation of the trial region. This figure provides a visual
representation of the results in
(\citeproc{ref-tab-result_resp_type}{\textbf{tab-result\_resp\_type?}}),
illustrating the individual observations (trials) across both
dimensions.

Figure \textbf{?@fig-rest\_type\_all} visualizes the reasons why the
estimated profits of the B2 group are significantly lower than those of
the A2 group. The figures in the first row illustrate how the YMSR can
lead to profit losses when compared to thecEOSR. In the right figure of
the first row, the estimated profit response curve is flatter than the
yield response curve, highlighting how a flatter profit curve
contributes to potential profit losses at the YMSR. In the Type B2
group, the yield response to seeding follows a quadratic pattern,
plateauing beyond the YMSR. However, the profit response to seeding
diminishes rapidly beyond the YMSR, which explains the potential loss of
profit due to excessive seeding. Additionally, in several B2 trials, the
yield response is not a quadratic plateau but a quadratic diminishing
curve. As a result, profits decrease more sharply with higher seeding
rates in these fields. In the Type A1 and B1 groups, as shown in the
figures in the bottom row, many trials have their EOSRs at either the
minimum or maximum of the trial input range.

\section{Discussion}\label{discussion}

The tables and figures in the results section present the estimated
yield and profit based on a fixed crop and seed price ratio of \$5.5 per
bushel of corn and \$3.8 per thousand seeds. To further evaluate
farmers' seeding decisions across different response types under varying
crop and seed price ratios, estimated yield and profit were also
calculated based on the annual price ratio changes over the past 10
years.

Figure \textbf{?@fig-dif\_proseed\_by\_price} shows the estimated profit
for each response type under the highest relative seed cost (top) and
the lowest relative seed cost (bottom). When comparing these figures
with Figure \textbf{?@fig-dif\_pro\_seed\_by\_comb} , more trials fall
into the A1 and A2 categories under the lowest relative seed cost.
However, even with lower relative seed costs, the estimated profit loss
due to excessive seeding rates (to the right of the black dashed
vertical line) remains higher than the loss from deficient seeding
rates.

\section{Conclusion}\label{conclusion}

This study evaluates the estimated profit of farmers' SQSR and EOSR in
corn production across the U.S. Midwest corn belt, focusing on potential
profit losses from over- or under-seeding behavior of farmers. The
analysis, based on 100 on-farm trials conducted from 2016 to 2023,
reveals that many farmers tend to over-seed, choosing seeding rates
(SQSR) higher than the estimated EOSR. The median SQSR is 34K seeds per
acre, compared to an EOSR of 30.2K, resulting in a yield difference of
3.4 bu/ac and an average profit loss of \$18.8 to \$69.3 per acre
depending on the response type.

Farmers' higher-than-optimal seeding rates, driven by past
recommendations or their own experience, are often not adjusted to
current economic and environmental conditions. Despite variations in
yield responses, excessive seeding consistently leads to diminishing
returns due to rising seed costs, which have approached fertilizer costs
in recent decades. The findings highlight that reducing seeding rates in
fields with high estimated profit losses can significantly enhance
profitability, especially as seed prices continue to rise. By better
understanding the yield response to seeding rates and adjusting
practices accordingly, farmers can minimize unnecessary input costs and
maximize profit, particularly under increasingly frequent droughts and
high-temperature events.

\section*{References}\label{references}
\addcontentsline{toc}{section}{References}

\phantomsection\label{refs}
\begin{CSLReferences}{1}{0}
\bibitem[\citeproctext]{ref-assefa2018analysis}
Assefa, Yared, Carter, Paul, Hinds, Mark, Bhalla, Gaurav, Schon, Ryan,
Jeschke, Mark, Paszkiewicz, Steve, Smith, Stephen, and Ciampitti,
Ignacio A, {``Analysis of long term study indicates both agronomic
optimal plant density and increase maize yield per plant contributed to
yield gain,''} \emph{Scientific Reports}, 8 (2018), 4937 (Nature
Publishing Group UK London).

\bibitem[\citeproctext]{ref-bullock2019data}
Bullock, David S, Boerngen, Maria, Tao, Haiying, Maxwell, Bruce, Luck,
Joe D, Shiratsuchi, Luciano, Puntel, Laila, and Martin, Nicolas F,
{``The data-intensive farm management project: Changing agronomic
research through on-farm precision experimentation,''} \emph{Agronomy
Journal}, 111 (2019), 2736--2746 (Wiley Online Library).

\bibitem[\citeproctext]{ref-bullock2009value}
Bullock, David S, Ruffo, Matı́as L, Bullock, Donald G, and Bollero,
Germán A, {``The value of variable rate technology: An
information-theoretic approach,''} \emph{American Journal of
Agricultural Economics}, 91 (2009), 209--223 (Wiley Online Library).

\bibitem[\citeproctext]{ref-edge2024processing}
Edge, Brittani, Mieno, Taro, and Bullock, David S, {``Processing of
on-farm precision experiment data in the DIFM project,''} \emph{Center
for Open Science}, (2024).

\bibitem[\citeproctext]{ref-fernandez2014genetically}
Fernandez-Cornejo, Jorge, Wechsler, Seth, Livingston, Mike, and
Mitchell, Lorraine, {``Genetically engineered crops in the united
states,''} \emph{USDA-ERS Economic Research Report}, (2014).

\bibitem[\citeproctext]{ref-kukal2018climate}
Kukal, Meetpal S, and Irmak, Suat, {``Climate-driven crop yield and
yield variability and climate change impacts on the US great plains
agricultural production,''} \emph{Scientific reports}, 8 (2018), 1--18
(Nature Publishing Group).

\bibitem[\citeproctext]{ref-lacasa2020bayesian}
Lacasa, Josefina, Gaspar, Adam, Hinds, Mark, Jayasinghege Don, Sampath,
Berning, Dan, and Ciampitti, Ignacio A, {``Bayesian approach for maize
yield response to plant density from both agronomic and economic
viewpoints in north america,''} \emph{Scientific reports}, 10 (2020),
15948 (Nature Publishing Group UK London).

\bibitem[\citeproctext]{ref-lix2021design}
Li, Xiaofei, Taro Mieno, and Bullock, David S, {``The economic
performances of different trial designs in on-farm precision
experimentation: A monte carlo evaluation,''} \emph{Working paper},
(2021).

\bibitem[\citeproctext]{ref-licht2017corn}
Licht, Mark A, Lenssen, Andrew W, and Elmore, Roger W, {``Corn (zea mays
l.) seeding rate optimization in iowa, USA,''} \emph{Precision
Agriculture}, 18 (2017), 452--469 (Springer).

\bibitem[\citeproctext]{ref-lindsey2018modeling}
Lindsey, Alexander J, Thomison, Peter R, and Nafziger, Emerson D,
{``Modeling the effect of varied and fixed seeding rates at a small-plot
scale,''} \emph{Agronomy Journal}, 110 (2018), 2456--2461 (Wiley Online
Library).

\bibitem[\citeproctext]{ref-morris2018strengths}
Morris, Thomas F, Murrell, T Scott, Beegle, Douglas B, Camberato, James
J, Ferguson, Richard B, Grove, John, Ketterings, Quirine, Kyveryga,
Peter M, Laboski, Carrie AM, McGrath, Joshua M, and others, {``Strengths
and limitations of nitrogen rate recommendations for corn and
opportunities for improvement,''} \emph{Agronomy journal}, 110 (2018),
1--37 (Wiley Online Library).

\bibitem[\citeproctext]{ref-illinois2023seed}
Nafziger, Emerson D, and Fontes, Giovani Preza, {``Planting corn in
2023,''} \emph{Department of Crop Sciences, University of Illinois},
(2023).

\bibitem[\citeproctext]{ref-usdanass}
NASS, USDA, {``Census of agriculture, crop production historical track
records,''} \emph{www.nass.usda.gov/AgCensus}, (2024).

\bibitem[\citeproctext]{ref-nielsen2019yield}
Nielsen, RL, Lee, Jason, Hettinga, John, and Camberato, Jim, {``Yield
response of corn to plant population in indiana,''} \emph{Purdue
University Department of Agronomy Applied Crop Production Research
Update}, (2019).

\bibitem[\citeproctext]{ref-rigden2020combined}
Rigden, AJ, Mueller, ND, Holbrook, NM, Pillai, Natesh, and Huybers,
Peter, {``Combined influence of soil moisture and atmospheric
evaporative demand is important for accurately predicting US maize
yields,''} \emph{Nature Food}, 1 (2020), 127--133 (Nature Publishing
Group UK London).

\bibitem[\citeproctext]{ref-saavoss2021trends}
Saavoss, Monica, Capehart, Tom, McBride, William, and Effland, Anne,
{``Trends in production practices and costs of the US corn sector,''}
\emph{10.22004/ag.econ.312954}, (2021).

\bibitem[\citeproctext]{ref-r2021language}
Team, R Core, and others, {``A language and environment for statistical
computing,''} \emph{https://www.R-project.org/}, (2021).

\bibitem[\citeproctext]{ref-thornton2022daymet}
Thornton, MM, Shrestha, R, Wei, Y, Thornton, PE, Kao, S, and Wilson, BE,
{``Daymet: Monthly climate summaries on a 1-km grid for north america,
version 4 R1. ORNL DAAC, oak ridge, tennessee, USA.''}
\emph{https://doi.org/10.3334/ORNLDAAC/2131}, (2022).

\bibitem[\citeproctext]{ref-wood2017generalized}
Wood, Simon N, {``Generalized additive models: An introduction with
r,''} (chapman; hall/CRC, 2017).

\end{CSLReferences}




\end{document}
